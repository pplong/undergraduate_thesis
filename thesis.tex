\documentclass[senior,final,11pt]{iscs-thesis}

\etitle{A Method of Improving QoS:  Explorations of the Possibility of Function Combinations in Service Compositions}
\jtitle{品質改良のアプローチ:機能のぶれを許すサービス合成}
%
\eauthor{Ziyuan Wang}
\jauthor{王子源}
\esupervisor{Shinichi Honiden}
\jsupervisor{本位田真一}
\supervisortitle{Professor} % Professor, etc.
\date{February 9, 2016}
%-------------------
\begin{document}
\begin{eabstract}
There is a growing need for web service providers to develop customised and flexible web services as quick as they can. One way to satisfy this demand is to utilise service compositions, which provide a method of consolidating several services to a richer service. Because of the uncertainty in the feasibility of the functional and QoS requirements, it is not guaranteed that service compositions succeed with suitable solutions.

One way to satisfy the given requirements is to control the given two kinds of requirements. There have been studies on methods of optimising the QoS with fixed functional requirements. However, the search space in that case is limited, which would possibly not include service compositions with better QoS whose functions are slightly different from the required one. Therefore, I focus on the possibility of improving the QoS by compromising on functions or exploring the possibility of function combinations. This enables better advice of service compositions to users. In this article, I propose a method of efficiently searching different possibilities to support users to balance functional requirements and QoS requirements.
\end{eabstract}
\begin{jabstract}
近年,ウェブサービスプロバイダーにとって,ユーザーの好みに合わせた,柔軟なウェブサービスを短時間で作る必要性が増してきている.そのために,よく使われている手法として,複数のサービスを一つにまとめてより高機能なサービスを作るサービス合成と呼ばれる手法がある.サービス合成において,入力として受け取る機能や品質に関する要求を必ず満たせるとは限らないという問題がある.

機能と品質の両方を達成するために既存研究で提案されている手法として,その片方を固定して保証し,もう片方を最適化するというものがある.しかし,この手法では,機能を固定するために,探索空間が限られているため,
要求された機能に妥協を許した場合に品質が大きく高まるような合成を見逃す可能性がある.よって,本論文では,ユーザーが高品質の合成を見つけることがより容易になるよう,機能を妥協し,要求された機能と少しのぶれを許すことで探索空間を増やすことによるQoSの最適化の可能性について議論した.本論文では,ユーザーが機能要求と品質要求のバランスを取れるように効率的に様々な合成の可能性を探索する手法を提案した.

\end{jabstract}
\maketitle

\begin{acknowledge}
I appreciate Prof. Ishikawa's help in improving the structure of the paper.
\end{acknowledge}

\frontmatter 
\tableofcontents
%\listoffigures
%\listoftables 
%\lstlistoflistings
%-------------------
\mainmatter 

\chapter{INTREODUCTION}
Web service, service-based system, and service composition are significant.
%ws 基于service的系\UTF{7EDF} 和service合成是很重要的
A lot of existing studies on QoS-aware service composition.
%介\UTF{7ECD}下上面提到的几个名\UTF{8BCD},\UTF{8BA8}\UTF{8BBA}下它\UTF{4EEC}的来\UTF{9F99}去脉 
Fixed functional requirements -> may lose potential improvement of QoS, difficulty for users to properly define the functional requirements because of uncertainty of the feasibility of requirements given available services
%\UTF{8BA8}\UTF{8BBA}它\UTF{4EEC}的不好,固定fun\UTF{5BFC}致没法找到更好的qos
Tis paper ... (summary of the remainder).
%本文的主\UTF{9898},干了什\UTF{4E48}。
Chapter 2 describes ... Chapter 3...
%\UTF{6BCF}一章做了什\UTF{4E48}%
\section{Background}
%背景
\section{Goals}
%我\UTF{4EEC}的目的
\section{Contributions}
%本文的\UTF{8D21}献
\cite{4065825}. 
%-------------------
\chapter{Preliminary}%1-18------------------------
%已有的定\UTF{4E49}%
Existing definitions.

\section{Service}
%所\UTF{8C13}service,就是功能可重\UTF{590D}利用的
A service S is a reusable system that provides functionalities which are documented in a service description. This description defines a 5-tuple \em{(S.I, S.O, S.P, S.E, S.Q)} where \em{S.I, S.O} are abbreviated from the required input and output paremeteres of the S, S.P and S.E are the preconditions and effects which means the necessary condition of utilising S and the consequence after running S, and S.Q is the set of the QoS attributes of S. 
service (instance) and service task
IOPEs -> modelled only with IO in this paper
+ 
Q
\section{QoS}
multi-criteria
normalized

\section{Functional Compliance}
Connectivity between two services

def: strong/weak

\section{Workflow}
Workflow templates: sequence of service tasks
Workflow: sequence of services (service instances)

Functional requirements: ぶれなし (existing)

\section{QoS optimization}
this time single-objective

%-------------------
\chapter{Extended Composition Problem}: Proposal1

motivation again

%ぶれありのcomposition problem定義
%def. 何段階ずれてもよい
(actually, not so meaningful to use too weak services)


\chapter{Algorithm for Extended Composition Problem}: Proposal 2

%%% Design 1
\subsection{Functionality Graph}
From [Wagner'11]
def.
algorithm

\subsection{Skyline}
def.
algorithm

\subsection{Naive Algorithm}
%for all 機能のぶれ (same, IN_one_weak-OUT_same, IN_same-OUT_one_weak, ...)
 % solve

\subsection{Algorithm}
% ぶれの探索順序をソート -> ランダム,強い順,弱い順
%for all 機能のぶれ (same, IN_one_weak-OUT_same, IN_same-OUT_one_weak, ...)
 % if (これまでに求めた最適解が,今回も使えるなら)
 %   それをreturn
 % solve(前回の探索結果) // さらに再利用?

\chapter{Experiments}

\section{Implementation}
Implemented with Python.

\section{Extended Composition Problem}
Evaluation of Proposal 1
How QoS can be improved by allowing compromise in functional requirements?

\section{Algorithm}
Evaluation of Proposal 2
How fast can the proposed algorithm solve the extended problem?
- with skyline vs. without skyline
- with skyline vs. proposed algorithm with skyline


\chapter{Discussion}

% related work can come here

\subsection{Proposal}
The extended problem should be significant because in the experiment

The proposed algorithm made good improvement because in the experiment 

% if difficult, combine to the experiment chapter

\subsection{Future Work}
- QoS constraints
- adaptive


\chapter{Conclusion}

%-------------------
\bibliographystyle{plain} 
\bibliography{myref} 
%-------------------
\end{document}
